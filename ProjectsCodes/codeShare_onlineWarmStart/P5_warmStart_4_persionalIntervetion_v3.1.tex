%% LyX 2.1.5 created this file.  For more info, see http://www.lyx.org/.
%% Do not edit unless you really know what you are doing.
\documentclass[12pt,twoside,UTF8,winfonts]{extarticle}
\renewcommand{\rmdefault}{ptm}
\usepackage[T1]{fontenc}
\usepackage[utf8]{inputenc}
\usepackage{geometry}
\geometry{verbose,tmargin=2.5cm,bmargin=2.5cm,lmargin=2.25cm,rmargin=2.25cm}
\usepackage{color}
\definecolor{page_backgroundcolor}{rgb}{1, 1, 1}
\pagecolor{page_backgroundcolor}
\usepackage{bm}
\usepackage{amsmath}
\usepackage{amssymb}
\usepackage[unicode=true,pdfusetitle,
 bookmarks=true,bookmarksnumbered=false,bookmarksopen=false,
 breaklinks=false,pdfborder={0 0 0},backref=false,colorlinks=true]
 {hyperref}
\hypersetup{
 citecolor=blue, urlcolor=blue, linkcolor=blue}
\usepackage{breakurl}

\makeatletter
%%%%%%%%%%%%%%%%%%%%%%%%%%%%%% Textclass specific LaTeX commands.
\ifx\proof\undefined
\newenvironment{proof}[1][\protect\proofname]{\par
\normalfont\topsep6\p@\@plus6\p@\relax
\trivlist
\itemindent\parindent
\item[\hskip\labelsep\scshape #1]\ignorespaces
}{%
\endtrivlist\@endpefalse
}
\providecommand{\proofname}{Proof}
\fi

%%%%%%%%%%%%%%%%%%%%%%%%%%%%%% User specified LaTeX commands.
% 如果没有这一句命令,XeTeX会出错,原因参见
% http://bbs.ctex.org/viewthread.php?tid=60547
\DeclareRobustCommand\nobreakspace{\leavevmode\nobreak\ }

\usepackage{epsfig}
\usepackage{multirow}
\usepackage{algorithm}
\usepackage{algorithmic}% 
\usepackage{graphicx} 
\usepackage{url}

%\usepackage{ctex}
%\usepackage{times}
%\setmainfont{Times New Roman}

\usepackage{mdwlist}
\usepackage{enumitem}
\setenumerate[1]{itemsep=1pt,partopsep=1pt,parsep=1pt,topsep=1pt}
\setitemize[1]{itemsep=1pt,partopsep=1pt,parsep=1pt,topsep=1pt}
\setdescription{itemsep=1pt,partopsep=1pt,parsep=1pt,topsep=1pt}

\usepackage{amsmath}
%\usepackage{amsthm}
\def\proof{\noindent{\bf  Proof}: } % 使 proof 粗体
\def\QEDclosed{\mbox{\rule[0pt]{1.3ex}{1.3ex}}}  \def\endproof{\hspace*{\fill}~\QEDclosed\par\endtrivlist\unskip}

\makeatother

\begin{document}
\global\long\def\mtbfA{\mathbf{A}}
 \global\long\def\mtbfa{\mathbf{a}}
 \global\long\def\mebfA{\bar{\mtbfA}}
 \global\long\def\mebfa{\bar{\mtbfa}}


\global\long\def\mhbfA{\widehat{\mathbf{A}}}
 \global\long\def\mhbfa{\widehat{\mathbf{a}}}
 \global\long\def\mtcalA{\mathcal{A}}
 \global\long\def\mtbbA{\mathbb{A}}
 

\global\long\def\mtbfB{\mathbf{B}}
 \global\long\def\mtbfb{\mathbf{b}}
 \global\long\def\mebfB{\bar{\mtbfB}}
 \global\long\def\mebfb{\bar{\mtbfb}}
 

\global\long\def\mhbfB{\widehat{\mathbf{B}}}
 \global\long\def\mhbfb{\widehat{\mathbf{b}}}
 \global\long\def\mtcalB{\mathcal{B}}
 \global\long\def\mtbbB{\mathbb{B}}
 

\global\long\def\mtbfC{\mathbf{C}}
 \global\long\def\mtbfc{\mathbf{c}}
 \global\long\def\mebfC{\bar{\mtbfC}}
 \global\long\def\mebfc{\bar{\mtbfc}}


\global\long\def\mhbfC{\widehat{\mathbf{C}}}
 \global\long\def\mhbfc{\widehat{\mathbf{c}}}
 \global\long\def\mtcalC{\mathcal{C}}
 \global\long\def\mtbbC{\mathbb{C}}
 

\global\long\def\mtbfD{\mathbf{D}}
 \global\long\def\mtbfd{\mathbf{d}}
 \global\long\def\mebfD{\bar{\mtbfD}}
 \global\long\def\mebfd{\bar{\mtbfd}}
 

\global\long\def\mhbfD{\widehat{\mathbf{D}}}
 \global\long\def\mhbfd{\widehat{\mathbf{d}}}
 \global\long\def\mtcalD{\mathcal{D}}
 \global\long\def\mtbbD{\mathbb{D}}
 

\global\long\def\mtbfE{\mathbf{E}}
 \global\long\def\mtbfe{\mathbf{e}}
 \global\long\def\mebfE{\bar{\mtbfE}}
 \global\long\def\mebfe{\bar{\mtbfe}}


\global\long\def\mhbfE{\widehat{\mathbf{E}}}
 \global\long\def\mhbfe{\widehat{\mathbf{e}}}
 \global\long\def\mtcalE{\mathcal{E}}
 \global\long\def\mtbbE{\mathbb{E}}
 

\global\long\def\mtbfF{\mathbf{F}}
 \global\long\def\mtbff{\mathbf{f}}
 \global\long\def\mebfF{\bar{\mathbf{F}}}
 \global\long\def\mebff{\bar{\mathbf{f}}}
 

\global\long\def\mhbfF{\widehat{\mathbf{F}}}
 \global\long\def\mhbff{\widehat{\mathbf{f}}}
 \global\long\def\mtcalF{\mathcal{F}}
 \global\long\def\mtbbF{\mathbb{F}}
 

\global\long\def\mtbfG{\mathbf{G}}
 \global\long\def\mtbfg{\mathbf{g}}
 \global\long\def\mebfG{\bar{\mathbf{G}}}
 \global\long\def\mebfg{\bar{\mathbf{g}}}


\global\long\def\mhbfG{\widehat{\mathbf{G}}}
 \global\long\def\mhbfg{\widehat{\mathbf{g}}}
 \global\long\def\mtcalG{\mathcal{G}}
 \global\long\def\mtbbG{\mathbb{G}}
 

\global\long\def\mtbfH{\mathbf{H}}
 \global\long\def\mtbfh{\mathbf{h}}
 \global\long\def\mebfH{\bar{\mathbf{H}}}
 \global\long\def\mebfh{\bar{\mathbf{h}}}


\global\long\def\mhbfH{\widehat{\mathbf{H}}}
 \global\long\def\mhbfh{\widehat{\mathbf{h}}}
 \global\long\def\mtcalH{\mathcal{H}}
 \global\long\def\mtbbH{\mathbb{H}}
 

\global\long\def\mtbfI{\mathbf{I}}
 \global\long\def\mtbfi{\mathbf{i}}
 \global\long\def\mebfI{\bar{\mathbf{I}}}
 \global\long\def\mebfi{\bar{\mathbf{i}}}


\global\long\def\mhbfI{\widehat{\mathbf{I}}}
 \global\long\def\mhbfi{\widehat{\mathbf{i}}}
 \global\long\def\mtcalI{\mathcal{I}}
 \global\long\def\mtbbI{\mathbb{I}}
 

\global\long\def\mtbfJ{\mathbf{J}}
 \global\long\def\mtbfj{\mathbf{j}}
 \global\long\def\mebfJ{\bar{\mathbf{J}}}
 \global\long\def\mebfj{\bar{\mathbf{j}}}


\global\long\def\mhbfJ{\widehat{\mathbf{J}}}
 \global\long\def\mhbfj{\widehat{\mathbf{j}}}
 \global\long\def\mtcalJ{\mathcal{J}}
 \global\long\def\mtbbJ{\mathbb{J}}
 

\global\long\def\mtbfK{\mathbf{K}}
 \global\long\def\mtbfk{\mathbf{k}}
 \global\long\def\mebfK{\bar{\mathbf{K}}}
 \global\long\def\mebfk{\bar{\mathbf{k}}}


\global\long\def\mhbfK{\widehat{\mathbf{K}}}
 \global\long\def\mhbfk{\widehat{\mathbf{k}}}
 \global\long\def\mtcalK{\mathcal{K}}
 \global\long\def\mtbbK{\mathbb{K}}
 

\global\long\def\mtbfL{\mathbf{L}}
 \global\long\def\mtbfl{\mathbf{l}}
 \global\long\def\mebfL{\bar{\mathbf{L}}}
 \global\long\def\mebfl{\bar{\mathbf{l}}}
 

\global\long\def\mhbfL{\widehat{\mathbf{K}}}
 \global\long\def\mhbfl{\widehat{\mathbf{k}}}
 \global\long\def\mtcalL{\mathcal{L}}
 \global\long\def\mtbbL{\mathbb{L}}
 

\global\long\def\mtbfM{\mathbf{M}}
 \global\long\def\mtbfm{\mathbf{m}}
 \global\long\def\mebfM{\bar{\mathbf{M}}}
 \global\long\def\mebfm{\bar{\mathbf{m}}}


\global\long\def\mhbfM{\widehat{\mathbf{M}}}
 \global\long\def\mhbfm{\widehat{\mathbf{m}}}
 \global\long\def\mtcalM{\mathcal{M}}
 \global\long\def\mtbbM{\mathbb{M}}
 

\global\long\def\mtbfN{\mathbf{N}}
 \global\long\def\mtbfn{\mathbf{n}}
 \global\long\def\mebfN{\bar{\mathbf{N}}}
 \global\long\def\mebfn{\bar{\mathbf{n}}}
 

\global\long\def\mhbfN{\widehat{\mathbf{N}}}
 \global\long\def\mhbfn{\widehat{\mathbf{n}}}
 \global\long\def\mtcalN{\mathcal{N}}
 \global\long\def\mtbbN{\mathbb{N}}
 

\global\long\def\mtbfO{\mathbf{O}}
 \global\long\def\mtbfo{\mathbf{o}}
 \global\long\def\mebfO{\bar{\mathbf{O}}}
 \global\long\def\mebfo{\bar{\mathbf{o}}}


\global\long\def\mhbfO{\widehat{\mathbf{O}}}
 \global\long\def\mhbfo{\widehat{\mathbf{o}}}
 \global\long\def\mtcalO{\mathcal{O}}
 \global\long\def\mtbbO{\mathbb{O}}
 

\global\long\def\mtbfP{\mathbf{P}}
 \global\long\def\mtbfp{\mathbf{p}}
 \global\long\def\mebfP{\bar{\mathbf{P}}}
 \global\long\def\mebfp{\bar{\mathbf{p}}}


\global\long\def\mhbfP{\widehat{\mathbf{P}}}
 \global\long\def\mhbfp{\widehat{\mathbf{p}}}
 \global\long\def\mtcalP{\mathcal{P}}
 \global\long\def\mtbbP{\mathbb{P}}
 

\global\long\def\mtbfQ{\mathbf{Q}}
 \global\long\def\mtbfq{\mathbf{q}}
 \global\long\def\mebfQ{\bar{\mathbf{Q}}}
 \global\long\def\mebfq{\bar{\mathbf{q}}}


\global\long\def\mhbfQ{\widehat{\mathbf{Q}}}
 \global\long\def\mhbfq{\widehat{\mathbf{q}}}
\global\long\def\mtcalQ{\mathcal{Q}}
 \global\long\def\mtbbQ{\mathbb{Q}}
 

\global\long\def\mtbfR{\mathbf{R}}
 \global\long\def\mtbfr{\mathbf{r}}
 \global\long\def\mebfR{\bar{\mathbf{R}}}
 \global\long\def\mebfr{\bar{\mathbf{r}}}


\global\long\def\mhbfR{\widehat{\mathbf{R}}}
 \global\long\def\mhbfr{\widehat{\mathbf{r}}}
\global\long\def\mtcalR{\mathcal{R}}
 \global\long\def\mtbbR{\mathbb{R}}


\global\long\def\mtbfS{\mathbf{S}}
 \global\long\def\mtbfs{\mathbf{s}}
 \global\long\def\mebfS{\bar{\mathbf{S}}}
 \global\long\def\mebfs{\bar{\mathbf{s}}}
 

\global\long\def\mhbfS{\widehat{\mathbf{S}}}
 \global\long\def\mhbfs{\widehat{\mathbf{s}}}
\global\long\def\mtcalS{\mathcal{S}}
 \global\long\def\mtbbS{\mathbb{S}}
 

\global\long\def\mtbfT{\mathbf{T}}
 \global\long\def\mtbft{\mathbf{t}}
 \global\long\def\mebfT{\bar{\mathbf{T}}}
 \global\long\def\mebft{\bar{\mathbf{t}}}


\global\long\def\mhbfT{\widehat{\mathbf{T}}}
 \global\long\def\mhbft{\widehat{\mathbf{t}}}
 \global\long\def\mtcalT{\mathcal{T}}
 \global\long\def\mtbbT{\mathbb{T}}
 

\global\long\def\mtbfU{\mathbf{U}}
 \global\long\def\mtbfu{\mathbf{u}}
 \global\long\def\mebfU{\bar{\mathbf{U}}}
 \global\long\def\mebfu{\bar{\mathbf{u}}}


\global\long\def\mhbfU{\widehat{\mathbf{U}}}
 \global\long\def\mhbfu{\widehat{\mathbf{u}}}
 \global\long\def\mtcalU{\mathcal{U}}
 \global\long\def\mtbbU{\mathbb{U}}
 

\global\long\def\mtbfV{\mathbf{V}}
 \global\long\def\mtbfv{\mathbf{v}}
 \global\long\def\mebfV{\bar{\mathbf{V}}}
 \global\long\def\mebfv{\bar{\mathbf{v}}}


\global\long\def\mhbfV{\widehat{\mathbf{V}}}
 \global\long\def\mhbfv{\widehat{\mathbf{v}}}
\global\long\def\mtcalV{\mathcal{V}}
 \global\long\def\mtbbV{\mathbb{V}}
 

\global\long\def\mtbfW{\mathbf{W}}
 \global\long\def\mtbfw{\mathbf{w}}
 \global\long\def\mebfW{\bar{\mathbf{W}}}
 \global\long\def\mebfw{\bar{\mathbf{w}}}


\global\long\def\mhbfW{\widehat{\mathbf{W}}}
 \global\long\def\mhbfw{\widehat{\mathbf{w}}}
 \global\long\def\mtcalW{\mathcal{W}}
 \global\long\def\mtbbW{\mathbb{W}}
 

\global\long\def\mtbfX{\mathbf{X}}
 \global\long\def\mtbfx{\mathbf{x}}
 \global\long\def\mebfX{\bar{\mtbfX}}
 \global\long\def\mebfx{\bar{\mtbfx}}


\global\long\def\mhbfX{\widehat{\mathbf{X}}}
 \global\long\def\mhbfx{\widehat{\mathbf{x}}}
 \global\long\def\mtcalX{\mathcal{X}}
 \global\long\def\mtbbX{\mathbb{X}}
 

\global\long\def\mtbfY{\mathbf{Y}}
 \global\long\def\mtbfy{\mathbf{y}}
\global\long\def\mebfY{\bar{\mathbf{Y}}}
 \global\long\def\mebfy{\bar{\mathbf{y}}}
 

\global\long\def\mhbfY{\widehat{\mathbf{Y}}}
 \global\long\def\mhbfy{\widehat{\mathbf{y}}}
 \global\long\def\mtcalY{\mathcal{Y}}
 \global\long\def\mtbbY{\mathbb{Y}}
 

\global\long\def\mtbfZ{\mathbf{Z}}
 \global\long\def\mtbfz{\mathbf{z}}
 \global\long\def\mebfZ{\bar{\mathbf{Z}}}
 \global\long\def\mebfz{\bar{\mathbf{z}}}
 

\global\long\def\mhbfZ{\widehat{\mathbf{Z}}}
 \global\long\def\mhbfz{\widehat{\mathbf{z}}}
\global\long\def\mtcalZ{\mathcal{Z}}
 \global\long\def\mtbbZ{\mathbb{Z}}
 

\global\long\def\mtth{\text{th}}


\global\long\def\mtbfzero{\mathbf{0}}
 \global\long\def\mtbfone{\mathbf{1}}


\global\long\def\mttrace{\text{Tr}}


\global\long\def\mttotalVariation{\text{TV}}


\global\long\def\mtexpect{\mathbb{E}}
 

\global\long\def\mtdet{\text{det}}


\global\long\def\mtvec{\mathbf{\text{vec}}}
 

\global\long\def\mtvar{\mathbf{\text{var}}}
 

\global\long\def\mtcov{\mathbf{\text{cov}}}
 

\global\long\def\mtsubTo{\mathbf{\text{s.t.}}}
 

\global\long\def\mtfor{\text{for}}


\global\long\def\mtrank{\text{rank}}
 

\global\long\def\mtrankn{\text{rankn}}
 

\global\long\def\mtdiag{\mathbf{\text{diag}}}
 

\global\long\def\mtsign{\mathbf{\text{sign}}}
 

\global\long\def\mtloss{\mathbf{\text{loss}}}
 

\global\long\def\mtwhen{\text{when}}
 

\global\long\def\mtwhere{\text{where}}
 

\global\long\def\mtif{\text{if}}
 


\title{The influence of warm start on the online personalized RL method
for the mobile health intervention}


\author{Feiyun ZHU}

\maketitle
{\small{}\tableofcontents{}}{\small \par}


\paragraph{Abstract}

Online personalized reinforcement learning (RL) is more and more popular
for the mobile health intervention. It can personalize the type, mode
and dose of intervention according to user's ongoing status and needs.
However from the perspective of the methodology, the online RL has
two drawbacks: 1) it does not have a good initialization of model
parameters; 2) at the beginning of the online learning, the RL methods
have too few samples to support the actor-critic RL updates, which
easily leads to some sub-optimal local minimal that does not perform
well. Those two drawbacks both delay the online learning process before
achieving good results, which is highly expensive in time costs and
risky for the peoples involved to abandon the mobile health intervention. 

To address the above two problems, we propose a new online RL methodology
that make full use of the data accumulated in the former study as
well as the knowledge obtained previously. We hope to highly accelerate
the online learning process. Besides, we design an experiment to verify
which of the two improvement really makes a difference. Experiment
results show dramatically improvement has been obtained, especially
at the beginning of the online RL. 

\begin{flushleft}
Specifically in this draft, we have three mobile health learning schemes: 
\par\end{flushleft}
\begin{enumerate}
\item The batch RL. For this learning scheme, we generate the (state, action
and immediate reward) trajectory via the micro random trials, where
the intervention is given in the probability of 50\%. Given all these
data, we use the batch actor-cirtc reinforcement learning algorithm
to learn the policy. Specifically, the critic update is via the LSTDQ;
the actor update is via the fmincon. 
\item The 2nd scheme is the personalized online learning, where the RL algorithm
interacts with the environment and improves in the process. This method
usually needs a warm start. One straight way is to collect a small
trajectory of data via micro random trial; and then starts the online
learning. However, this method greatly delay the learning process,
which is very costly in the sense of time involved in the study. That
is, it needs to randomly run the study, to collect a set of data and
then to start the learning. 
\item The 3rd schemes deal with the problem of the 2nd method. It makes
use of the data from the 1st study and treat the learnt parameter
from the 1st study as a warm study for the online RL of the new users.
This scheme is supposed to learn much faster than the 2nd methods 
\end{enumerate}

\paragraph{Notations}
\begin{enumerate}
\item We use the upper case bold Roman letters to denote matrices and lower
case bold Roman letters to represent vectors. For a matrix $\mathbf{Y}=\left[Y_{ln}\right]\in\mathbb{R}^{L\times N}$,
we denote its $l^{\text{th}}$ row and $n^{\text{th}}$ column via
$\mathbf{y}^{l}$ and $\mathbf{y}_{n}$ respectively. We use the uppercase
letter to denote the random variable and the lower case to represent
the corresponding realization.
\item \begin{flushleft}
$S_{i}\in\mathbb{R}^{Lo}$ is the random state at the $i$-th decision
point, and $s_{i}$ is the corresponding realization. 
\par\end{flushleft}
\item $f\left(S_{i}\right)\in\mathbb{R}^{Lf}$ is the value feature via
the basic functions. In this draft, we have three choices: 1) the
state vector itself, 2) the polynomial basic function and 3) the radial
basis function (RBF).
\item $\mathbf{x}\left(f\left(S_{i}\right),A_{i}\right)\in\mathbb{R}^{Lh}$
is a vector function to achieve the feature for the linear value approximate.
The definition is as follows 
\[
\mathbf{x}\left(f\left(S_{i}\right),A_{i}\right)=\left[1,f\left(S_{i}\right)^{T},A_{i},A_{i}f\left(S_{i}\right)^{T}\right]^{T}\in\mathbb{R}^{Lh=2\times Lf+2}.
\]
 In the following, we use $\mathbf{x}\left(S_{i},A_{i}\right)$ or
$\mathbf{x}_{i}$ to denote $\mathbf{x}\left(f\left(S_{i}\right),A_{i}\right)$
for simple. 
\item $\mathbf{z}\left(S_{i}\right)\in\mathbb{R}^{Lg}$ is a vector function
to construct the policy feature. Specifically, we use the state variables
directly along with a constant 1, i.e. $\mathbf{z}\left(S_{i}\right)=\left[S_{i}^{T},1\right]^{T}\in\mathbb{R}^{Lo+1}$.
For the sake of convenience, we usually use $\mathbf{z}_{i}\in\mathbb{R}^{Lg}$
to represent $\mathbf{z}\left(S_{i}\right)\in\mathbb{R}^{Lg}$.
\item ${\displaystyle \pi_{\theta}\left(a\mid s\right)=\frac{\exp\left(a\mathbf{z}\left(s\right)^{T}\theta\right)}{1+\exp\left(\mathbf{z}\left(s\right)^{T}\theta\right)}}$
is the policy model, where $\theta$ is the model parameter and $a\in\left\{ 0,1\right\} $
is the binary action.
\item $N$ denotes the number of people involved in the study; $Ls$ is
the number of entries in the state vector $S_{i}$; $Lf$ is the feature
length of $f\left(S_{i}\right)$ after basic function; $Lh$ is the
feature length of $\mathbf{x}\left(f\left(S_{i}\right),A_{i}\right),$
which is the feature for the linear value approximate; $Lg$ denotes
the length of policy feature. 
\end{enumerate}

\paragraph{Organization of the rest of the paper}

The rest of this paper is organized as follows: in Section\ \ref{sec:Batch_RLs},
the batch reinforcement learning methods are introduced. Those method
make use of the data generated via the micro-randomed trials. That
is, they use the data ready there; they do not need to generate the
trajectory in the online learning process. In Section\ \ref{sec:online_RLs_noWS},
the algorithms for the regular online RL methods, without warm starts,
are provided. The algorithms for the online RL methods with warm start
are given in Section\ \ref{sec:online_RLs_WS}. The details on the
simulation data, the performance metric as well as the value features
are given in the Section\ \ref{sec:experiments_Data_evaluateMetric_feature}.


\section{Algorithm for the \textcolor{blue}{batch} actor-critic RL \label{sec:Batch_RLs}}


\subsection{Data Collection via the Micro-Randomized Trials \label{sec:DataCollectionViaMicro-randomedTrials}}
\begin{enumerate}
\item Section\ \ref{sub:Datasets} introduces the details on the data generation
(simulation) model (i.e. the MDP model), which is used to generate
the MDP trajectories. Specifically, we need to set the following parameters 

\begin{enumerate}
\item $N$ the number of people involved in the data set. 
\item $T$ the trajectory length; 
\item $\sigma_{r}$ and $\sigma_{s}$ are the noise strength when simulating
the reward and the state respectively.
\item $\sigma_{b}$ is the variance ofthe Gaussian noise put in the basic
$\bm{\beta}$ to make each individual's MDP different from each other.
The value of $\sigma_{b}$ indicates how different the people are.
\end{enumerate}
\item Draw the trajectories for all the $N$ people; each trajectory includes
$T$ time points. Specifically at each time point and for each individual
in the trajectory, we have a sample tuple, including four elements:
the current state, action, reward and next-state as follows
\begin{equation}
\mathcal{D}=\left\{ \mathcal{D}_{n}\right\} _{n=1}^{N},\qquad\text{where}\ \mathcal{D}_{n}=\left\{ \left(s_{i},a_{i},r_{i},s_{i}^{'}\right)\mid i=1,2,\cdots,T\right\} .\label{eq:trajectory4allPeople_set-1}
\end{equation}
where the action is randomly chosen with probability of 50\% to give
the intervention. So we totally have $NT=N\times T$ time points drawn
from all the $N$ individuals. Collecting all the data, we have the
following sample matrices 
\begin{align}
\mtbfX & =\left[\mtbfX_{1},\mtbfX_{2},\cdots,\mtbfX_{N}\right]\in\mathcal{\mathbb{R}}^{L\times NT}\nonumber \\
\mathbf{\mtbfY} & =\left[\mtbfY_{1},\mtbfY_{2},\cdots,\mtbfY_{N}\right]\in\mathcal{\mathbb{R}}^{L\times NT}\label{eq:trajectory4allPeople_matrix-1}\\
\mtbfr & =\left[\mtbfr_{1}^{T},\mtbfr_{2}^{T},\cdots,\mtbfr_{N}^{T}\right]^{T}\in\mathcal{\mathbb{R}}^{NT},\nonumber 
\end{align}
where the $n$-th individual's data is collected in $\left\{ \mtbfX_{n},\mtbfY_{n},\mtbfr_{n}\right\} $
as follows\textbf{\emph{ }}
\begin{align*}
\mtbfX_{n} & =\left[\mathbf{x}\left(s_{1},a_{1}\right),\mathbf{x}\left(s_{2},a_{2}\right)\cdots,\mathbf{x}\left(s_{T},a_{T}\right)\right]\in\mathbb{R}^{L\times T}\\
\mtbfY_{n} & =\left[\sum_{a}\mathbf{x}\left(s_{1}^{'},a\right)\pi_{\theta}\left(a\mid s_{1}^{'}\right),\cdots,\sum_{a}\mathbf{x}\left(s_{T}^{'},a\right)\pi_{\theta}\left(a\mid s_{T}^{'}\right)\right]\in\mathbb{R}^{L\times T}\\
\mtbfr_{n} & =\left[r_{1},r_{2},\cdots,r_{T}\right]^{T}\in\mathbb{R}^{T},
\end{align*}
where $\mtbfX_{n}$ collects the value feature for all the current
point, $\mtbfY_{n}$ has all value feature for the next point, and
$\mtbfr_{n}$ contains all the immediate reward. For simplicity, we
use $\mathbf{x}_{n}=\mathbf{x}\left(s_{n},a_{n}\right)$ to denote
the feature vector at the $n^{\text{th}}$ time point for the value
approximate model here and in the rest of this draft. 
\end{enumerate}

\subsection{Algorithm for the batch discount reward \& contextual bandit via
the LSTDQ \& Fmincon}


\subsubsection{Algorithm}
\begin{enumerate}
\item Generate the trajectories for $N$ people, each with the length of
$T$, and organize them based Section\ \ref{sec:DataCollectionViaMicro-randomedTrials}.
\item \textbf{Input} \textbf{parameters} for the discount reward method
and the contextual bandit method.

\begin{enumerate}
\item $\zeta_{c}$ is the strength of $\ell_{2}$-constraint on $\mathbf{w}$,
which is the parameter for the value function.
\item $\zeta_{a}$ is the strength of $\ell_{2}$-constraint on $\theta$,
which is the parameter for the policy function. 
\item $\gamma\in\left[0,1\right]$ is a known \textbf{\emph{discount factor}}
(under assumptions does not depend on $t$). To get the contextual
bandit method, we simply set $\gamma=0.$
\end{enumerate}
\item \textbf{Repeat}

\begin{enumerate}
\item \textbf{\emph{Critic update to get}}\emph{ the optimal $\widehat{\mathbf{w}}_{0}$
as follows }
\[
\widehat{\mathbf{w}}_{0}=\left(\zeta_{c}\mathbf{I}+\frac{1}{NT}\sum_{i=1}^{NT}\mathbf{x}_{i}\left(\mathbf{x}_{i}-\gamma\mathbf{y}_{i+1}\right)^{T}\right)^{-1}\left(\frac{1}{NT}\sum_{i=1}^{NT}\mathbf{x}_{i}r_{i}\right),
\]
where $\mathbf{x}_{i}=\mathbf{x}\left(f\left(S_{i}\right),A_{i}\right)$
is the feature vector at decision point $i$ for the value funciton;
$\mathbf{y}_{i+1}=\sum_{a}\mathbf{x}\left(f\left(S_{i+1}\right),a\right)\pi_{\theta}\left(a\mid S_{i+1}\right)$
is the feature at the next time point; $r_{i}$ is the immediate reward
at the $i$-th point; $\zeta_{c}$ is the balancing parameter for
the $\ell_{2}$ constraint to avoid singular solver. Using the notations
in Eq.\ (\ref{eq:trajectory4allPeople_matrix-1}), we have a compact
critic update solution 
\[
\widehat{\mathbf{w}}_{0}=\left[\zeta_{c}\mathbf{I}+\frac{1}{NT}\mtbfX\left(\mtbfX-\gamma\mtbfY\right)^{T}\right]^{-1}\left(\frac{1}{NT}\mtbfX\mathbf{r}\right).
\]

\item \textbf{\emph{Actor update}}\textbf{ }to get\textbf{ ${\displaystyle \hat{\theta}_{0}=\arg\max_{\theta}}\ \hat{J}(\theta,\widehat{\mathbf{w}}_{0})$},
which is equivalent to
\begin{equation}
\min_{\theta}\mathcal{J}\left(\theta\right)=-\frac{1}{NT}\sum_{i=1}^{NT}\sum_{a\in\left\{ 0,1\right\} }Q\left(S_{i},a;\widehat{\mathbf{w}}_{0}\right)\cdot\pi_{\theta}\left(a|S_{i}\right)+\frac{\zeta_{a}}{2}\left\Vert \theta\right\Vert _{2}^{2},\label{eq:actorObj_discountRwd-1}
\end{equation}
where $\zeta_{a}$ is a balancing parameter for the $\ell_{2}$ constraint
on $\theta$ to prevent over-fitting; $Q\left(S_{i},a;\widehat{\mathbf{w}}_{0}\right)=\mathbf{x}\left(S_{i},a\right)^{T}\widehat{\mathbf{w}}_{0}$
is the current estimated value. 
\end{enumerate}

\textbf{Until }convergence


\textbf{Output:} the final parameter $\hat{\theta}_{0}$ in the policy
function and $\widehat{\mathbf{w}}_{0}$ in the value approximate. 

\end{enumerate}

\subsubsection{Appendix: the gradient for\ (\ref{eq:actorObj_discountRwd-1}) \label{sub:AppendixGradientOfDiscountActorUpdate}}

The gradient for\ (\ref{eq:actorObj_discountRwd-1}) is as follows
\[
\nabla_{\theta}\mathcal{J}\left(\theta\right)=\frac{1}{NT}\sum_{i=1}^{NT}\sum_{a\in\left\{ 0,1\right\} }Q\left(S_{i},a;\widehat{\mathbf{w}}_{0}\right)\cdot\nabla_{\theta}\pi_{\theta}\left(a|S_{i}\right)+\zeta_{a}\theta
\]
where $\pi_{\theta}\left(a|S_{i}\right)=\frac{\exp\left(a\theta^{T}\mathbf{z}_{i}\right)}{1+\exp\left(\theta^{T}\mathbf{z}_{i}\right)}$,
$\mathbf{z}_{i}=\mathbf{z}\left(S_{i}\right)$ is the feature for
policy, and 
\begin{align*}
\nabla_{\theta}\pi_{\theta}\left(a|S_{i}\right) & =\frac{\left(\nabla_{\theta}\exp\left(a\theta^{T}\mathbf{z}\right)\right)\left(1+\exp\left(\theta^{T}\mathbf{z}\right)\right)-\left(\nabla_{\theta}\left(1+\exp\left(\theta^{T}\mathbf{z}\right)\right)\right)\exp\left(a\theta^{T}\mathbf{z}\right)}{\left(1+\exp\left(\theta^{T}\mathbf{z}\right)\right)^{2}}.\\
{\color{red}} & =\frac{\exp\left(a\theta^{T}\mathbf{z}\right)\left(a\mathbf{z}\right)\left(1+\exp\left(\theta^{T}\mathbf{z}\right)\right)-\exp\left(\theta^{T}\mathbf{z}\right)\mathbf{z}\exp\left(a\theta^{T}\mathbf{z}\right)}{\left(1+\exp\left(\theta^{T}\mathbf{z}\right)\right)^{2}}\\
{\color{red}} & =\frac{\exp\left(a\theta^{T}\mathbf{z}\right)}{1+\exp\left(\theta^{T}\mathbf{z}\right)}\cdot\mathbf{z}\frac{a\left(1+\exp\left(\theta^{T}\mathbf{z}\right)\right)-\exp\left(\theta^{T}\mathbf{z}\right)}{1+\exp\left(\theta^{T}\mathbf{z}\right)}\\
{\color{red}} & =\pi_{\theta}\left(a\mid S_{i}\right)\mathbf{z}\left(a-\pi_{\theta}\left(1\mid S_{i}\right)\right).
\end{align*}
For $a\in\left\{ 0,1\right\} $, we have 
\begin{align*}
\nabla_{\theta}\pi_{\theta}\left(0|\mathbf{z}\right) & =\frac{-\exp\left(\theta^{T}\mathbf{z}\right)\mathbf{z}}{\left(1+\exp\left(\theta^{T}\mathbf{z}\right)\right)^{2}}\\
\nabla_{\theta}\pi_{\theta}\left(1|\mathbf{z}\right) & =\frac{\exp\left(\theta^{T}\mathbf{z}\right)\mathbf{z}}{\left(1+\exp\left(\theta^{T}\mathbf{z}\right)\right)^{2}}=-\nabla_{\theta}\pi_{\theta}\left(0|\mathbf{z}\right).
\end{align*}



\subsection{Algorithm for the batch average reward via the LSTDQ \& Fmincon}


\subsubsection{Algorithm}
\begin{enumerate}
\item Generate the trajectories for $N$ people, each with the trajectory
length as $T$, and organize them based on Section\ \ref{sec:DataCollectionViaMicro-randomedTrials}.
\item \textbf{Input} \textbf{parameters} for the discount and contextual
bandit algorithm

\begin{enumerate}
\item $\zeta_{c}$ is the strength of $\ell_{2}$-constraint on $\mathbf{w}$,
which is the parameter for the value function.
\item $\zeta_{a}$ is the strength of $\ell_{2}$-constraint on $\theta$,
which is the parameter for the policy function. 
\end{enumerate}
\item \textbf{Repeat}

\begin{enumerate}
\item \textbf{\emph{Critic update to get}}\emph{ the optimal $\widehat{\mathbf{w}}_{0}$
as follows}
\begin{equation}
\mhbfw_{0}=\left(\zeta_{c}\mtbfI+\frac{1}{NT}\sum_{i=1}^{NT}\left(\mtbfx_{i}-\mebfx_{0}\right)\left(\mtbfx_{i}-\gamma\mtbfy_{i+1}\right)^{T}\right)^{-1}\left(\frac{1}{NT}\sum_{i=1}^{NT}\left(\mtbfx_{i}-\mebfx_{0}\right)r_{i}\right),\label{eq:AvgRwd_criticUpdate}
\end{equation}
where $\mathbf{x}_{i}=\mathbf{x}\left(f\left(S_{i}\right),A_{i}\right)$
is the feature vector at decision point $i$ for the value funciton;
$\mathbf{y}_{i+1}=\sum_{a}\mathbf{x}\left(f\left(S_{i+1}\right),a\right)\pi_{\theta}\left(a\mid S_{i+1}\right)$
is the feature at the next time point; $r_{i}$ is the immediate reward
at the $i$-th point; $\zeta_{c}$ is the balancing parameter for
the $\ell_{2}$ constraint to avoid singular solver; $\mebfx_{0}=\frac{1}{NT}\sum_{i=1}^{NT}\mtbfx_{i}$
is the mean of all the current value features. 
\item \textbf{\emph{Actor update}}\textbf{ }to get\textbf{ ${\displaystyle \hat{\theta}_{0}=\arg\max_{\theta}}\ \hat{J}(\theta,\widehat{\mathbf{w}}_{0})$},
which is equivalent to\textbf{
\begin{alignat}{1}
\min_{\theta}\mtcalJ_{t}\left(\theta\right) & =-\frac{1}{NT}\sum_{i=1}^{NT}\left(r_{i}+V\left(S_{i+1};\mhbfw_{0}\right)-Q\left(S_{i},A_{i};\mhbfw_{0}\right)\right)+\frac{\zeta_{a}}{2}\left\Vert \theta\right\Vert _{2}^{2},\label{eq:obj_AvgRwd_actorUpdate}\\
 & =-\frac{1}{NT}\sum_{i=1}^{NT}\left(r_{i}+\sum_{a\in\mtcalA}Q\left(S_{i+1},a;\mhbfw_{0}\right)\pi_{\theta}\left(a|S_{i+1}\right)-Q\left(S_{i},A_{i};\mhbfw_{0}\right)\right)+\frac{\zeta_{a}}{2}\left\Vert \theta\right\Vert _{2}^{2}.\nonumber 
\end{alignat}
}Here the actor update and the critic update are done separately;
besides, we substitute the value of $\mhbfw_{0}$ instead of the expression
of $\mhbfw_{0}$ into the actor objective function. $Q\left(S_{i},A_{i};\mhbfw_{0}\right)$
is not related to the parameter $\theta.$ Removing the irrelevant
terms with $\theta$ in\ (\ref{eq:obj_AvgRwd_actorUpdate}), we have
the new objective function
\begin{equation}
\min_{\theta}\mathcal{J}\left(\theta\right)=-\frac{1}{NT}\sum_{i=1}^{NT}\sum_{a\in\mtcalA}Q\left(S_{i+1},a;\widehat{\mathbf{w}}_{0}\right)\cdot\pi_{\theta}\left(a|S_{i+1}\right)+\frac{\zeta_{a}}{2}\left\Vert \theta\right\Vert _{2}^{2},\label{eq:actorObj_discountRwd_2}
\end{equation}
where $\zeta_{a}$ is a balancing parameter for the $\ell_{2}$ constraint
on $\theta$ to prevent over-fitting; $Q\left(S_{i+1},a;\widehat{\mathbf{w}}_{0}\right)=\mathbf{x}\left(S_{i+1},a\right)^{T}\widehat{\mathbf{w}}_{0}$
is the current estimated value. 
\end{enumerate}

\textbf{Until }convergence


\textbf{Output:} the final parameter $\hat{\theta}_{0}$ in the policy
function and $\widehat{\mathbf{w}}_{0}$ in the value approximate. 

\end{enumerate}

\subsubsection{Appendixe\#1: prove of the critic update\ (\ref{eq:AvgRwd_criticUpdate})}
\begin{proof}
According to Susan's draft, the basic equation for the actor critic
average reward is
\begin{equation}
0=\sum_{i=1}^{t}\left(r_{i}-\eta+V\left(S_{i+1};\mtbfw\right)-Q\left(S_{i},A_{i};\mtbfw\right)\right)\left(\begin{array}{c}
1\\
f\left(S_{i}\right)\left(1-A_{i}\right)\\
f\left(S_{i}\right)A_{i}
\end{array}\right)-\zeta_{a}\mtbfI\left(\begin{array}{c}
0\\
\mtbfw_{0}\\
\mtbfw_{1}
\end{array}\right)\label{eq:AvgRwdBasicFunction-1}
\end{equation}
where $v=\left(v_{0}^{T},v_{1}^{T}\right)^{T}$ and $\eta\in\mtbbR$
are unknown variables; $Q\left(S_{i},A_{i};\mtbfw\right)=\mtbfw^{T}\mtbfx_{i}$
and $V\left(S_{i+1};\mtbfw\right)=\mtbfw^{T}\mtbfy_{i+1}$. If we
using a more general feature, like $\mtbfx_{i}=\mtbfx\left(f\left(S_{i}\right),A_{i}\right)$
and $\mathbf{y}_{i+1}=\sum_{a}\mathbf{x}\left(f\left(S_{i+1}\right),a\right)\pi_{\theta}\left(a\mid S_{i+1}\right)$,
(\ref{eq:AvgRwdBasicFunction-1}) is turned into 
\begin{equation}
0=\sum_{i=1}^{t}\left(r_{i}-\eta+\mtbfy_{i+1}^{T}\mtbfw-\mtbfx_{i}^{T}\mtbfw\right)\left(\begin{array}{c}
1\\
\mtbfx_{i}
\end{array}\right)-\zeta_{a}\mtbfI\left(\begin{array}{c}
0\\
\mtbfw
\end{array}\right)\label{eq:AvgRwdBasicFunction_new-1}
\end{equation}
There are two rows in\ (\ref{eq:AvgRwdBasicFunction_new-1}). Considering
the first row, we have 
\[
0=\sum_{i=1}^{t}\left(r_{i}-\eta+\left(\mtbfy_{i+1}-\mtbfx_{i}\right)^{T}\mtbfw\right)
\]
 and \textcolor{blue}{
\begin{equation}
\eta=\frac{1}{t}\sum_{i=1}^{t}\left(r_{i}+\left(\mtbfy_{i+1}-\mtbfx_{i}\right)^{T}\mtbfw\right).\label{eq:AvgRwd_update_eta-1}
\end{equation}
}Considering the last row in\ (\ref{eq:AvgRwdBasicFunction_new-1}),
we have 
\[
\sum_{i=1}^{t}\mtbfx_{i}\left(r_{i}-\eta+\left(\mtbfy_{i+1}-\mtbfx_{i}\right)^{T}\mtbfw\right)-\zeta_{a}\mtbfw=0
\]
 and \textcolor{blue}{
\begin{equation}
\left(\zeta_{a}\mtbfI+\sum_{i=1}^{t}\mtbfx_{i}\left(\mtbfx_{i}-\mtbfy_{i+1}\right)^{T}\right)\mtbfw=\sum_{i=1}^{t}\mtbfx_{i}\left(r_{i}-\eta\right).\label{eq:AvgRwd_update_v-1}
\end{equation}
}Substituting the estimator of $\eta$ in\ (\ref{eq:AvgRwd_update_eta-1})
into\ (\ref{eq:AvgRwd_update_v-1}), we have
\begin{align*}
 & \left(\zeta_{a}\mtbfI+\sum_{i=1}^{t}\mtbfx_{i}\left(\mtbfx_{i}-\mtbfy_{i+1}\right)^{T}\right)\mtbfw=\sum_{i=1}^{t}\mtbfx_{i}\left(r_{i}-\eta\right)\\
= & \sum_{i=1}^{t}\mtbfx_{i}\left(r_{i}-\frac{1}{t}\sum_{i=1}^{t}\left(r_{i}+\left(\mtbfy_{i+1}-\mtbfx_{i}\right)^{T}\mtbfw\right)\right)\\
= & \sum_{i=1}^{t}\mtbfx_{i}r_{i}-\left[\sum_{i=1}^{t}\mtbfx_{i}\cdot\frac{1}{t}\sum_{i=1}^{t}\left(r_{i}+\left(\mtbfy_{i+1}-\mtbfx_{i}\right)^{T}\mtbfw\right)\right]\\
= & \sum_{i=1}^{t}\mtbfx_{i}r_{i}-\left(\frac{1}{t}\sum_{i=1}^{t}\mtbfx_{i}\right)\left(\sum_{i=1}^{t}\left(r_{i}+\left(\mtbfy_{i+1}-\mtbfx_{i}\right)^{T}\mtbfw\right)\right).
\end{align*}
Let $\mebfx_{t}=\left(\frac{1}{t}\sum_{i=1}^{t}\mtbfx_{i}\right)$,
we have the following derivations 
\begin{align*}
\left(\zeta_{a}\mtbfI+\sum_{i=1}^{t}\mtbfx_{i}\left(\mtbfx_{i}-\mtbfy_{i+1}\right)^{T}\right)\mtbfw= & \sum_{i=1}^{t}\mtbfx_{i}r_{i}-\mebfx_{t}\left(\sum_{i=1}^{t}\left(r_{i}+\left(\mtbfy_{i+1}-\mtbfx_{i}\right)^{T}\mtbfw\right)\right)\\
= & \sum_{i=1}^{t}\left(\mtbfx_{i}-\mebfx_{t}\right)r_{i}+\sum_{i=1}^{t}\mebfx_{t}\left(\mtbfx_{i}-\mtbfy_{i+1}\right)^{T}\mtbfw,
\end{align*}
and
\[
\left(\zeta_{a}\mtbfI+\sum_{i=1}^{t}\left(\mtbfx_{i}-\mebfx_{t}\right)\left(\mtbfx_{i}-\mtbfy_{i+1}\right)^{T}\right)\mtbfw=\sum_{i=1}^{t}\left(\mtbfx_{i}-\mebfx_{t}\right)r_{i}.
\]

\end{proof}

\subsubsection{Appendix\#2: the gradient for\ (\ref{eq:actorObj_discountRwd-1})}

Please check Section\ \ref{sub:AppendixGradientOfDiscountActorUpdate}
for detail.


\section{Algorithm for the \textcolor{blue}{online} actor-critic RL without
Warm-Start \label{sec:online_RLs_noWS}}


\subsection{Online actor-cirtc Contextual bandit\label{sub:Alg_4_onlineBandit_noWS}}
\begin{enumerate}
\item \textbf{Input}:

\begin{enumerate}
\item $T_{\text{max}}$ the maximum trajectory length in the online learning
setting. 
\item $\zeta_{c}$ is the strength of $\ell_{2}$-constraint on $v$, which
is the parameter for the value function.
\item $\zeta_{a}$ the strength of $\ell_{2}$-constraint on $\theta$,
which is the parameter for the policy function. 
\item $\mathbf{B}_{0}=\zeta_{c}\mathbf{I}_{Lh\times Lh}\in\mathbb{R}^{Lh\times Lh}$
and $\mathbf{a}_{0}=\mtbfzero\in\mathbb{R}^{Lh}$.
\end{enumerate}
\item \textbf{While} ( $t<T_{\text{max}}$ )

\begin{enumerate}
\item Observe context $s_{t}$. 
\item Draw an action $a_{t}$ according to the probability distribution
$\pi_{\theta}(a|s_{t}).$
\item Observe an immediate reward $r_{t}$; 
\item \textbf{\emph{Critic update to get}}\emph{ the optimal $\widehat{\mathbf{w}}_{t}$
at point} $t$ \emph{via
\begin{equation}
\widehat{\mathbf{w}}_{t}=\mathbf{B}_{t}^{-1}\mathbf{a}_{t},\label{eq:emmyOnlineBantidCriticUpdate-1}
\end{equation}
}where $\mathbf{B}_{t}=\mathbf{B}_{t-1}+\mathbf{x}\left(s_{t},a_{t}\right)\mathbf{x}\left(s_{t},a_{t}\right)^{T}$
and $\mathbf{a}_{t}=\mathbf{a}_{t-1}+\mathbf{x}\left(s_{t},a_{t}\right)r_{t}$.
Since $\left\{ \mathbf{B}_{t}\right\} _{t=1}^{T}$ is generally invertible,
we may use the \href{https://en.wikipedia.org/wiki/Sherman\%E2\%80\%93Morrison_formula}{Sherman–Morrison formula}
to highly reduce the computational cost of\ (\ref{eq:emmyOnlineBantidCriticUpdate-1}),
resulting in the following updates 
\[
\widehat{\mathbf{w}}_{t}=\widehat{\mathbf{w}}_{t-1}+\mathbf{B}_{t-1}^{-1}\mathbf{x}\left(s_{t},a_{t}\right)\left\{ \frac{\mathbf{x}\left(s_{t},a_{t}\right)^{T}\mathbf{B}_{t-1}^{-1}-r_{t}}{1+\mathbf{x}\left(s_{t},a_{t}\right)^{T}\mathbf{B}_{t-1}^{-1}\mathbf{x}\left(s_{t},a_{t}\right)}\right\} 
\]

\item \textbf{\emph{Actor update}}\textbf{ }to get\textbf{ ${\displaystyle \hat{\theta}_{t}=\arg\max_{\theta}}\ \hat{J}_{t}(\theta,\widehat{\mathbf{w}}_{t})$},
which is equivalent to 
\[
\min_{\theta}\mathcal{J}_{t}\left(\theta\right)=-\frac{1}{t}\sum_{i=1}^{t}\sum_{a}Q\left(S_{i},a;\mathbf{w}_{t}\right)\pi_{\theta}\left(a|S_{i}\right)+\frac{\zeta_{a}}{2}\left\Vert \theta\right\Vert _{2}^{2},
\]
where $\zeta_{a}$ is a balancing parameter for the $\ell_{2}$ constraint
on $\theta$ to prevent over-fitting; $Q\left(S_{i},a;\widehat{\mathbf{w}}_{t}\right)=\mathbf{x}\left(S_{i},a\right)^{T}\widehat{\mathbf{w}}_{t}$
is the current estimated value.
\end{enumerate}
\item \textbf{End While}
\item \textbf{Output:} the final parameter $\hat{\theta}_{t}$ in the policy
function\textbf{ }and $\mhbfw_{t}$ in the value approximate. 
\end{enumerate}

\subsection{Online actor-cirtc discount reward\label{sub:Alg_4_onlineDiscount_noWS}}
\begin{enumerate}
\item \textbf{Input}:

\begin{enumerate}
\item $T_{\text{max}}$ the maximum trajectory length in the online learning
setting. 
\item $\zeta_{c}$ is the strength of $\ell_{2}$-constraint on $v$, which
is the parameter for the value function.
\item $\zeta_{a}$ the strength of $\ell_{2}$-constraint on $\theta$,
which is the parameter for the policy function. 
\item $\gamma\in\left[0,1\right]$ is a known \textbf{\emph{discount factor}}
(under assumptions does not depend on $t$). To get the contextual
bandit method, we simply set $\gamma=0.$
\end{enumerate}
\item \textbf{While} ( $t<T_{\text{max}}$ )

\begin{enumerate}
\item Observe context $s_{t}$. 
\item Draw an action $a_{t}$ according to the probability distribution
$\pi_{\theta}(a|s_{t}).$
\item Observe an immediate reward $r_{t}$; 
\item \textbf{\emph{Critic update to get}}\emph{ the optimal $\widehat{\mathbf{w}}_{t}$
at point} $t$ \emph{via}
\[
\widehat{\mathbf{w}}_{t}=\left(\zeta_{c}\mathbf{I}+\frac{1}{t}\sum_{i=1}^{t}\mathbf{x}_{i}\left(\mathbf{x}_{i}-\gamma\mathbf{y}_{i+1}\right)^{T}\right)^{-1}\left(\frac{1}{t}\sum_{i=1}^{t}\mathbf{x}_{i}r_{i}\right),
\]
where $\mathbf{x}_{i}=\mathbf{x}\left(f\left(S_{i}\right),A_{i}\right)$
is the feature vector at decision point $i$ for the value funciton;
$\mathbf{y}_{i+1}=\sum_{a}\mathbf{x}\left(f\left(S_{i+1}\right),a\right)\pi_{\theta_{t}}\left(a\mid S_{i+1}\right)$
is the feature at the next time point; $r_{i}$ is the immediate reward
at the $i$-th point; $\zeta_{c}$ is the balancing parameter for
the $\ell_{2}$ constraint to avoid singular solver. Note that after
each actor update, we need to re-calculate the next value feature
$\mathbf{y}_{i+1}=\sum_{a}\mathbf{x}\left(f\left(S_{i+1}\right),a\right)\pi_{\theta_{t}}\left(a\mid S_{i+1}\right)$
based on the new obtained policy parameter $\theta_{t}.$
\item \textbf{\emph{Actor update}}\textbf{ }to get\textbf{ ${\displaystyle \hat{\theta}_{t}=\arg\max_{\theta}}\ \hat{J}_{t}(\theta,\widehat{\mathbf{w}}_{t})$},
which is equivalent to 
\[
\min_{\theta}\mathcal{J}_{t}\left(\theta\right)=-\frac{1}{t}\sum_{i=1}^{t}\sum_{a}Q\left(S_{i},a;\mathbf{w}_{t}\right)\pi_{\theta}\left(a|S_{i}\right)+\frac{\zeta_{a}}{2}\left\Vert \theta\right\Vert _{2}^{2},
\]
where $\zeta_{a}$ is a balancing parameter for the $\ell_{2}$ constraint
on $\theta$ to prevent over-fitting; $Q\left(S_{i},a;\widehat{\mathbf{w}}_{t}\right)=\mathbf{x}\left(S_{i},a\right)^{T}\widehat{\mathbf{w}}_{t}$
is the current estimated value.
\end{enumerate}
\item \textbf{End While}
\item \textbf{Output:} the final parameter $\hat{\theta}_{T}$ in the policy
function\textbf{ }and $\mhbfw_{t}$ in the value approximate. 
\end{enumerate}

\subsection{Online actor-cirtc Average Reward\label{sub:Alg_4_onlineAvgRwd_noWS}}
\begin{enumerate}
\item \textbf{Input}:

\begin{enumerate}
\item $T_{\text{max}}$ the maximum trajectory length in the online learning
setting. 
\item $\zeta_{c}$ is the strength of $\ell_{2}$-constraint on $v$, which
is the parameter for the value function.
\item $\zeta_{a}$ is the strength of $\ell_{2}$-constraint on $\theta$,
which is the parameter for the policy function. 
\end{enumerate}
\item \textbf{While} ( $t<T_{\text{max}}$ )

\begin{enumerate}
\item Observe context $s_{t}$. 
\item Draw an action $a_{t}$ according to the probability distribution
$\pi_{\theta}(a_{t}|s_{t}).$
\item Observe an immediate reward $r_{t}$.
\item \textbf{\emph{Critic update to get}}\emph{ the optimal $\widehat{\mathbf{w}}_{t}$
at point} $t$ \emph{via}
\[
\mhbfw_{t}=\left(\zeta_{c}\mtbfI+\frac{1}{t}\sum_{i=1}^{t}\left(\mtbfx_{i}-\mebfx_{0}\right)\left(\mtbfx_{i}-\gamma\mtbfy_{i+1}\right)^{T}\right)^{-1}\left(\sum_{i=1}^{t}\left(\mtbfx_{i}-\mebfx_{0}\right)r_{i}\right),
\]
where $\mathbf{x}_{i}=\mathbf{x}\left(f\left(S_{i}\right),A_{i}\right)$
is the feature vector at decision point $i$ for the value funciton;
$\mathbf{y}_{i+1}=\sum_{a}\mathbf{x}\left(f\left(S_{i+1}\right),a\right)\pi_{\theta}\left(a\mid S_{i+1}\right)$
is the feature at the next time point; $r_{i}$ is the immediate reward
at the $i$-th point; $\zeta_{c}$ is the balancing parameter for
the $\ell_{2}$ constraint to avoid singular solver. $\mebfx_{0}=\frac{1}{NT}\sum_{i=1}^{NT}\mtbfx_{i}$
is the mean of all the current value features. Note that after each
actor update, we need to re-calculate the next value feature $\mathbf{y}_{i+1}=\sum_{a}\mathbf{x}\left(f\left(S_{i+1}\right),a\right)\pi_{\theta_{t}}\left(a\mid S_{i+1}\right)$
based on the new obtained policy parameter $\theta_{t}.$
\item \textbf{\emph{Actor update}}\textbf{ }to get\textbf{ ${\displaystyle \hat{\theta}_{t}=\arg\max_{\theta}}\ \hat{J}_{t}(\theta,\widehat{\mathbf{w}}_{t})$},
which is equivalent to\textbf{
\begin{alignat*}{1}
\min_{\theta}\mtcalJ_{t}\left(\theta\right) & =-\frac{1}{t}\sum_{i=1}^{t}\left(r_{i}+\sum_{a\in\mtcalA}Q\left(S_{i+1},a;\mhbfw_{t}\right)\pi_{\theta}\left(a|S_{i+1}\right)-Q\left(S_{i},A_{i};\mhbfw_{t}\right)\right)+\frac{\zeta_{a}}{2}\left\Vert \theta\right\Vert _{2}^{2}.
\end{alignat*}
}Here the actor update and the critic update are done separately;
besides, we substitute the value of $\mhbfw_{t}$ instead of the expression
of $\mhbfw_{t}$ into the actor objective function. $Q\left(S_{i},A_{i};\mhbfw_{t}\right)$
is not related to the parameter $\theta.$ Removing the irrelevant
terms with $\theta$ in\ (\ref{eq:obj_AvgRwd_actorUpdate}), we have
the new objective function 
\[
\min_{\theta}\mathcal{J}_{t}\left(\theta\right)=-\frac{1}{t}\sum_{i=1}^{t}\sum_{a\in\mtcalA}Q\left(S_{i+1},a;\mhbfw_{t}\right)\cdot\pi_{\theta}\left(a|S_{i}\right)+\frac{\zeta_{a}}{2}\left\Vert \theta\right\Vert _{2}^{2},
\]
where $\zeta_{a}$ is a balancing parameter for the $\ell_{2}$ constraint
on $\theta$ to prevent over-fitting; $Q\left(S_{i},a;\widehat{\mathbf{w}}_{t}\right)=\mathbf{x}\left(S_{i},a\right)^{T}\widehat{\mathbf{w}}_{t}$
is the current estimated value.
\end{enumerate}
\item \textbf{End While}
\item \textbf{Output:} the final parameter $\hat{\theta}_{t}$ in the policy
function\textbf{ }and $\mathbf{\mhbfw}_{t}$ in the value approximate. 
\end{enumerate}

\section{Algorithms for the \textcolor{blue}{online} actor-cirtc RL with Warm-Start
\label{sec:online_RLs_WS}}

There are two kinds of differences between the RL methods with warm
start and those RL methods without warm-starts:
\begin{enumerate}
\item the RL with warm start make use of the data from the 1st set of $\beta$s
(i.e. individuals or MDPs), while the RL without warm start does not
make any use of the data from 1st set of $\beta$s. 
\item the RL with warm start takes the learnt value parameter and policy
parameter, via the batch RL methods on the 1st set of of data, as
the initialization of the parameters in the online learning. 
\end{enumerate}
In Sections\ \ref{sub:Alg_4_onlineBandit_WS},\ \ref{sub:Alg_4_onlineDiscount_WS}\ and\ \ref{sub:Alg_4_onlineAvgRwd_WS},
we will give the algorithms for the RL method with warm start. Note
that the terms in red uses the data generated via the micro-randomed
trial based on the 1st set of individuals. 


\paragraph*{Notations to differentiate the data from 1st set of individual and
the the data collected in the online process:}
\begin{itemize}
\item the varibles with bars are the data from the 1st set of beta, i.e.
$\left\{ \left(\mebfx_{n},\mebfy_{n},\bar{r}_{n}\right)\right\} _{n=1}^{NT},$
where $N$ is the number of people involved; $T$ is the trajectory
length. 
\item the normal varibles are the data generated in the online learning,
i.e. $\left\{ \left(\mtbfx_{i},\mtbfy_{i},r_{i}\right)\right\} _{i=1}^{t}$
, where $t$ is the current trajectory length in the online learning.
\end{itemize}

\subsection{Online actor-cirtc Contextual bandit with warm start \label{sub:Alg_4_onlineBandit_WS}}
\begin{enumerate}
\item \textbf{Input}:

\begin{enumerate}
\item $T_{\text{max}}$ the maximum trajectory length in the online learning
setting. 
\item $\zeta_{c}$ is the strength of $\ell_{2}$-constraint on $v$, which
is the parameter for the value function.
\item $\zeta_{a}$ the strength of $\ell_{2}$-constraint on $\theta$,
which is the parameter for the policy function. 
\item the learnt value parameter $\mhbfw_{0}$ and the learnt policy parameter
$\hat{\theta}_{0}$, via the batch RL methods on the data from the
1st set of individuals, as an initialization of the corresponding
parameters on the actor-critic RL methods. 
\end{enumerate}
\item \textbf{While} ( $t<T_{\text{max}}$ )

\begin{enumerate}
\item Observe context $s_{t}$. 
\item Draw an action $a_{t}$ according to the probability distribution
$\pi_{\theta}(a|s_{t}).$
\item Observe an immediate reward $r_{t}$; 
\item \textbf{\emph{Critic update to get}}\emph{ the optimal $\widehat{\mathbf{w}}_{t}$
at point} $t$ \emph{via
\[
\widehat{\mathbf{w}}_{t}=\left\{ \zeta_{c}\mathbf{I}+\frac{1}{t+1}\left[{\color{red}\frac{1}{NT}\sum_{j=1}^{NT}\mathbf{\mebfx}_{j}\mathbf{\mebfx}_{j}^{T}}+\sum_{i=1}^{t}\mathbf{x}_{i}\mathbf{x}_{i}^{T}\right]\right\} ^{-1}\left[\frac{1}{t+1}\left({\color{red}\frac{1}{NT}\sum_{j=1}^{NT}\mebfx_{j}\bar{r}_{j}}+\sum_{i=1}^{t}\mathbf{x}_{i}r_{i}\right)\right].
\]
}where $\left\{ \mebfx_{j}\right\} _{j=1}^{NT}$ is the data from
the 1st set of individuals (i.e. $N$ individuals, each is with a
trajectory of $T$ points); $\left(\mtbfx_{i}\right)_{i=1}^{t}$ is
the data that is collected when the RL method interacts with the MDP
system; $\mathbf{x}_{i}=\mathbf{x}\left(f\left(S_{i}\right),A_{i}\right)$
is the feature vector at decision point $i$ for the value funciton;
$\mathbf{y}_{i+1}=\sum_{a}\mathbf{x}\left(f\left(S_{i+1}\right),a\right)\pi_{\theta_{t}}\left(a\mid S_{i+1}\right)$
is the feature at the next time point; $r_{i}$ is the immediate reward
at the $i$-th point; $\zeta_{c}$ is the balancing parameter for
the $\ell_{2}$ constraint to avoid singular solver.
\item \textbf{\emph{Actor update}}\textbf{ }to get\textbf{ ${\displaystyle \hat{\theta}_{t}=\arg\max_{\theta}}\ \hat{J}_{t}(\theta,\widehat{\mathbf{w}}_{t})$},
which is equivalent to 
\[
\min_{\theta}\mathcal{J}\left(\theta\right)=-\frac{1}{t+1}\left\{ {\color{red}\frac{1}{NT}\sum_{j=1}^{NT}\sum_{a\in\left\{ 0,1\right\} }Q\left(\bar{S}_{j},a;\widehat{\mathbf{w}}_{t}\right)\pi_{\theta}\left(a|\bar{S}_{j}\right)}+\sum_{i=1}^{t}\sum_{a\in\left\{ 0,1\right\} }Q\left(S_{i},a;\widehat{\mathbf{w}}_{t}\right)\cdot\pi\left(a|S_{i}\right)\right\} +\frac{\zeta_{a}}{2}\left\Vert \theta\right\Vert _{2}^{2}.
\]
where $\zeta_{a}$ is a balancing parameter for the $\ell_{2}$ constraint
on $\theta$ to prevent over-fitting; $Q\left(S_{i},a;\widehat{\mathbf{w}}_{t}\right)=\mathbf{x}\left(S_{i},a\right)^{T}\widehat{\mathbf{w}}_{t}$
is the current estimated value.
\end{enumerate}
\item \textbf{End While}
\item \textbf{Output:} the final parameter $\hat{\theta}_{T}$ in the policy
function\textbf{ }and $\mathbf{w}_{t}$ in the value approximate. 
\end{enumerate}

\subsection{Online actor-cirtc discount reward with warm start \label{sub:Alg_4_onlineDiscount_WS}}
\begin{enumerate}
\item \textbf{Input}:

\begin{enumerate}
\item $T_{\text{max}}$ the maximum trajectory length in the online learning
setting. 
\item $\zeta_{c}$ is the strength of $\ell_{2}$-constraint on $v$, which
is the parameter for the value function.
\item $\zeta_{a}$ the strength of $\ell_{2}$-constraint on $\theta$,
which is the parameter for the policy function. 
\item $\gamma\in\left[0,1\right]$ is a known \textbf{\emph{discount factor}}
(under assumptions does not depend on $t$). To get the contextual
bandit method, we simply set $\gamma=0.$
\item the learnt value parameter $\mhbfw_{0}$ and the learnt policy parameter
$\hat{\theta}_{0}$, via the batch RL methods on the data from the
1st set of individuals, as an initialization of the corresponding
parameters on the actor-critic RL methods.
\end{enumerate}
\item \textbf{While} ( $t<T_{\text{max}}$ )

\begin{enumerate}
\item Observe context $s_{t}$. 
\item Draw an action $a_{t}$ according to the probability distribution
$\pi_{\theta}(a_{t}|s_{t}).$
\item Observe an immediate reward $r_{t}$.
\item \textbf{\emph{Critic update to get}}\emph{ the optimal $\widehat{\mathbf{w}}_{t}$
at point} $t$ \emph{via}
\[
\widehat{\mathbf{w}}_{t}=\left\{ \zeta_{c}\mathbf{I}+\frac{1}{t+1}\left[{\color{red}\frac{1}{NT}\sum_{j=1}^{NT}\mathbf{\mebfx}_{j}\left(\mathbf{\mebfx}_{j}-\gamma\mathbf{\mebfy}_{i+1}\right)^{T}}+\sum_{i=1}^{t}\mathbf{x}_{i}\left(\mathbf{x}_{i}-\gamma\mathbf{y}_{i+1}\right)^{T}\right]\right\} ^{-1}\left[\frac{1}{t+1}\left({\color{red}\frac{1}{NT}\sum_{j=1}^{NT}\mebfx_{j}\bar{r}_{j}}+\sum_{i=1}^{t}\mathbf{x}_{i}r_{i}\right)\right].
\]
where $\left\{ \mebfx_{j}\right\} _{j=1}^{NT}$ is the data from the
1st set of individuals (i.e. $N$ individuals, each is with a trajectory
of $T$ points); $\left(\mtbfx_{i}\right)_{i=1}^{t}$ is the data
that is collected when the RL method interacts with the MDP system;
$\mathbf{x}_{i}=\mathbf{x}\left(f\left(S_{i}\right),A_{i}\right)$
is the feature vector at decision point $i$ for the value funciton;
$\mathbf{y}_{i+1}=\sum_{a}\mathbf{x}\left(f\left(S_{i+1}\right),a\right)\pi_{\theta_{t}}\left(a\mid S_{i+1}\right)$
is the feature at the next time point; $r_{i}$ is the immediate reward
at the $i$-th point; $\zeta_{c}$ is the balancing parameter for
the $\ell_{2}$ constraint to avoid singular solver.
\item \textbf{\emph{Actor update}}\textbf{ }to get\textbf{ ${\displaystyle \hat{\theta}_{t}=\arg\max_{\theta}}\ \hat{J}_{t}(\theta,\widehat{\mathbf{w}}_{t})$},
which is equivalent to 
\[
\min_{\theta}\mathcal{J}\left(\theta\right)=-\frac{1}{t+1}\left\{ {\color{red}\frac{1}{NT}\sum_{j=1}^{NT}\sum_{a\in\left\{ 0,1\right\} }Q\left(\bar{S}_{j},a;\widehat{\mathbf{w}}_{t}\right)\pi_{\theta}\left(a|\bar{S}_{j}\right)}+\sum_{i=1}^{t}\sum_{a\in\left\{ 0,1\right\} }Q\left(S_{i},a;\widehat{\mathbf{w}}_{t}\right)\cdot\pi\left(a|S_{i}\right)\right\} +\frac{\zeta_{a}}{2}\left\Vert \theta\right\Vert _{2}^{2}.
\]
where $\zeta_{a}$ is a balancing parameter for the $\ell_{2}$ constraint
on $\theta$ to prevent over-fitting; $Q\left(S_{i},a;\widehat{\mathbf{w}}_{t}\right)=\mathbf{x}\left(S_{i},a\right)^{T}\widehat{\mathbf{w}}_{t}$
is the current estimated value.
\end{enumerate}
\item \textbf{End While}
\item \textbf{Output:} the final parameter $\hat{\theta}_{t}$ in the policy
function\textbf{ }and $\mathbf{w}_{t}$ in the value approximate. 
\end{enumerate}

\subsection{Online actor-cirtc Average Reward with Warm Start \label{sub:Alg_4_onlineAvgRwd_WS}}
\begin{enumerate}
\item \textbf{Input}:

\begin{enumerate}
\item $T_{\text{max}}$ the maximum trajectory length in the online learning
setting. 
\item $\zeta_{c}$ is the strength of $\ell_{2}$-constraint on $v$, which
is the parameter for the value function.
\item $\zeta_{a}$ the strength of $\ell_{2}$-constraint on $\theta$,
which is the parameter for the policy function. 
\item the learnt value parameter $\mhbfw_{0}$ and the learnt policy parameter
$\hat{\theta}_{0}$, via the batch RL methods on the data from the
1st set of individuals, as an initialization of the corresponding
parameters on the actor-critic RL methods.
\end{enumerate}
\item \textbf{While} ( $t<T_{\text{max}}$ )

\begin{enumerate}
\item Observe context $s_{t}$. 
\item Draw an action $a_{t}$ according to the probability distribution
$\pi_{\theta}(a|s_{t}).$
\item Observe an immediate reward $r_{t}$; 
\item \textbf{\emph{Critic update to get}}\emph{ the optimal $\widehat{\mathbf{w}}_{t}$
at point} $t$ \emph{via}
\begin{align*}
\mhbfw_{t}= & \left\{ \zeta_{c}\mtbfI+\frac{1}{t+1}\left[{\color{red}\frac{1}{NT}\sum_{j=1}^{NT}\left(\mathbf{\mebfx}_{j}-\mebfx_{0}\right)\left(\mathbf{\mebfx}_{j}-\gamma\mathbf{\mebfy}_{i+1}\right)^{T}}+\sum_{i=1}^{t}\left(\mtbfx_{i}-\mtbfx_{0}\right)\left(\mtbfx_{i}-\gamma\mtbfy_{i+1}\right)^{T}\right]\right\} ^{-1},\\
 & \left\{ \frac{1}{t+1}\left[{\color{red}\frac{1}{NT}\sum_{j=1}^{NT}\left(\mathbf{\mebfx}_{j}-\mebfx_{0}\right)r_{i}}+\sum_{i=1}^{t}\left(\mtbfx_{i}-\mtbfx_{0}\right)r_{i}\right]\right\} 
\end{align*}
where $\mathbf{x}_{i}=\mathbf{x}\left(f\left(S_{i}\right),A_{i}\right)$
is the feature vector at decision point $i$ for the value funciton;
$\mathbf{y}_{i+1}=\sum_{a}\mathbf{x}\left(f\left(S_{i+1}\right),a\right)\pi_{\theta}\left(a\mid S_{i+1}\right)$
is the feature at the next time point; $r_{i}$ is the immediate reward
at the $i$-th point; $\zeta_{c}$ is the balancing parameter for
the $\ell_{2}$ constraint to avoid singular solver. $\mtbfx_{0}=\frac{1}{t}\sum_{i=1}^{t}\mtbfx_{i}$
and $\mebfx_{0}=\frac{1}{NT}\sum_{i=1}^{NT}\mebfx_{i}$ are the means
of all the current value features. Note that after each actor update,
we need to re-calculate the next value feature $\mathbf{y}_{i+1}=\sum_{a}\mathbf{x}\left(f\left(S_{i+1}\right),a\right)\pi_{\theta_{t}}\left(a\mid S_{i+1}\right)$
based on the new obtained policy parameter $\theta_{t}.$
\item \textbf{\emph{Actor update}}\textbf{ }to get\textbf{ ${\displaystyle \hat{\theta}_{t}=\arg\max_{\theta}}\ \hat{J}_{t}(\theta,\widehat{\mathbf{w}}_{t})$},
which is equivalent to
\[
\min_{\theta}\mathcal{J}\left(\theta\right)=-\frac{1}{t+1}\left\{ {\color{red}\frac{1}{NT}\sum_{j=1}^{NT}\sum_{a\in\left\{ 0,1\right\} }Q\left(\bar{S}_{j+1},a;\widehat{\mathbf{w}}_{t}\right)\pi_{\theta}\left(a|\bar{S}_{j+1}\right)}+\sum_{i=1}^{t}\sum_{a\in\left\{ 0,1\right\} }Q\left(S_{i+1},a;\widehat{\mathbf{w}}_{t}\right)\cdot\pi_{\theta}\left(a|S_{i+1}\right)\right\} +\frac{\zeta_{a}}{2}\left\Vert \theta\right\Vert _{2}^{2}.
\]
where $\zeta_{a}$ is a balancing parameter for the $\ell_{2}$ constraint
on $\theta$ to prevent over-fitting; $Q\left(S_{i},a;\widehat{\mathbf{w}}_{t}\right)=\mathbf{x}\left(S_{i},a\right)^{T}\widehat{\mathbf{w}}_{t}$
is the current estimated value.
\end{enumerate}
\item \textbf{End While}
\item \textbf{Output:} the final parameter $\hat{\theta}_{t}$ in the policy
function\textbf{ }and $\mathbf{w}_{t}$ in the value approximate. 
\end{enumerate}

\section{The experiment settings \label{sec:experiments_Data_evaluateMetric_feature}}


\subsection{Datasets \label{sub:Datasets}}

In the experiments, a trajectory of $T$ (the trajectory length) tuples
for each individual is defined as follows 
\begin{equation}
\mathcal{D}_{T}=\left\{ \left(O_{0},A_{0},R_{0}\right),\left(O_{1},A_{1},R_{1}\right),\cdots,\left(O_{T},A_{T},R_{T}\right)\right\} ,\label{eq:jdkafdja-1}
\end{equation}
where the initial states and action are generated by $O_{0}\sim\mathrm{Normal}_{p_{1}}\left\lbrace 0,\Sigma\right\rbrace $
and $A_{0}=0$. Here we have $\Sigma=\left[\begin{array}{cc}
\Sigma_{1} & 0\\
0 & I_{p-3}
\end{array}\right]$ and ${\displaystyle \Sigma_{1}=\begin{bmatrix}1 & 0.3 & -0.3\\
0.3 & 1 & -0.3\\
-0.3 & -0.3 & 1
\end{bmatrix}}.$ For $t\geq1$, we have the state generation model as follows
\begin{align}
O_{t,1} & =\beta_{1}O_{t-1,1}+\xi_{t,1},\qquad\text{weather-high is good}\nonumber \\
O_{t,2} & =\beta_{2}O_{t-1,2}+\beta_{3}A_{t-1}+\xi_{t,2},\qquad\text{engagement}\label{eq:Dat=0000231_stateTrans_cmp3}\\
O_{t,3} & =\beta_{4}O_{t-1,3}+\beta_{5}O_{t-1,3}A_{t-1}+\beta_{6}A_{t-1}+\xi_{t,3},\quad\text{treatment fatigue}\nonumber \\
O_{t,j} & =\beta_{7}O_{t-1,j}+\xi_{t,j}.\quad j=4,\ldots,p\nonumber 
\end{align}
 The corresponding immediate reward model is as follows 
\begin{equation}
R_{t+1}=\beta_{14}\times\left[\beta_{8}+A_{t}\times(\beta_{9}+\beta_{10}O_{t,1}+\beta_{11}O_{t,2})+\beta_{12}O_{t,1}{\color{blue}-{\color{blue}\beta_{13}O_{t,3}}}+\varrho_{t,}\right],\label{eq:Dat=0000231_ImmediateRwd_cmp3}
\end{equation}
where ${\color{blue}-\beta_{13}O_{t,3}}$ is the treatment fatigue.
In the above generation model, the $\left\{ \xi_{t,i}\right\} _{i=1}^{p}$
at the $t$-th time point are the noise in the state transition model\ (\ref{eq:Dat=0000231_stateTrans_cmp3})
that is drawn from $\left\{ \xi_{t,i}\right\} _{i=1}^{p}\sim\textnormal{Normal}\left(0,\sigma_{s}^{2}\right)$.
$\varrho_{t}$ is the noise in the reward model\ (\ref{eq:Dat=0000231_ImmediateRwd_cmp3}),
which is defined as $\varrho_{t}\sim\textnormal{Normal}\left(0,\sigma_{r}^{2}\right)$.

To generate a group of $N$ people that is slightly different from
each other, we need $N$ slightly different MDPs. Formally, each MDP
is specified by the value of $\bm{\beta}$, cf. Eqn.\ (\ref{eq:Dat=0000231_stateTrans_cmp3})
and\ (\ref{eq:Dat=0000231_ImmediateRwd_cmp3}). The $\bm{\beta}s$
is generated the following two steps:
\begin{enumerate}
\item we set an initial (i.e. basic) $\bm{\beta}_{\text{basic}}=\left[0.40,0.25,0.35,0.65,0.10,0.50,0.22,2.00,0.15,0.20,0.32,0.10,0.45,5\right]$.
\item To make a group of $N$ people that is slight different, we set the
$\left\{ \bm{\beta}_{i}\right\} _{i=1}^{N}$ as 
\[
\bm{\beta}_{i}=\bm{\beta}_{\text{basic}}+\bm{\delta}_{i},\qquad\text{for}\ i\in\left\{ 1,2,\cdots,N\right\} ,
\]
where $\bm{\delta}_{i}\sim\text{Normal}\left(0,\sigma_{b}\mathbf{I}_{14}\right)$
and $\mathbf{I}_{14}\in\mathbb{R}^{14\times14}$ is an identity matrix.
\end{enumerate}
To generate the MDP for a future user, we will use this kind of method
to generate a new $\bm{\beta}.$ In this way, the $\bm{\beta}$ for
each individual is different from each other; $\sigma_{b}$ controls
how different the MDPs are, i.e. how different the individuals are.


\subsection{Evaluation Metrics for Quantitative Performance}

We want to see if the personalized RL methods could learn faster when
we initialize online algorithms with $\mathbf{B}_{0}$, $\mathbf{a}_{0}$
from the 1st data set. We want to compare them with not using $\mathbf{B}_{0}$,
$\mathbf{a}_{0}$ from the 1st data set, for different $t<T_{\max}$.
We will do these for a variety of $t$ not just $T_{\max}.$

We use one metric to measure the quality of the learnt policy, i.e.
the expectation of the long run average reward (i.e. ElrAR) $\mathbb{E}_{\pi}\left[\eta_{\hat{\theta}}\right]$,
where $\hat{\theta}$ is the learnt policy parameter. Intuitively,
ElrAR measures how much average reward in the long run we could get
by using the learnt policy $\hat{\theta}$. A larger ElrAR corresponds
to a better performance. Formally, there are three steps to get the
ElrAR:
\begin{itemize}
\item For each individual, a very long trajectory of $T=5,000$ is generated
by the using the policy parameter $\hat{\theta}$; 
\item By averaging the rewards for the last $4,000$ elements from the trajectory,
we could get the long run average reward 
\[
\eta_{\hat{\theta}}=\frac{1}{T-i}\sum_{j=i}^{T}R_{j+1}\left(S_{j},A_{j}\right)
\]
where $i=1000$, $T=5,000$. 
\item The expectation of the long run average reward is obtained by averaging
$\eta_{\hat{\theta}}$ among all the $N$ individual results the as
\[
\mathbb{E}_{\pi}\left[\eta_{\hat{\theta}}\right]\approx\frac{1}{N}\sum_{n=1}^{N}\eta_{\hat{\theta}}^{\left(n\right)},
\]
where $\eta_{\hat{\theta}}^{\left(n\right)}$ is the estimated long
run average reward for the $n^{\mtth}$ individual (i.e. MDP). 
\end{itemize}

\subsection{Feature construction \label{sub:FeatureConstruction}}

In this section, we introduce how to construct the features for the
policy function and the value approximate function respectively. The
policy feature is easy, which is simply adding one constant into the
state vector 
\[
\mathbf{z}\left(O_{i}\right)=\left[1,O_{t}\right]^{T}\in\mathbb{R}^{Lo+1}.
\]
The feature for the value approximation is more complex. It employs
the basic function 
\[
\mathbf{z}\left(O_{i},A_{i}\right)=\left[\left(1-A_{i}\right)f\left(O_{i}\right)^{T},A_{i}f\left(O_{i}\right)^{T}\right]^{T},
\]
where $f\left(O_{i}\right)$ is a basic function of state $O_{i}.$
In the follwoing, we will introduce two popular basic function for
the reinforcement learning\ \cite{George_2011_AAAI_ValueApproxViaFourier,Sutton_2012_Book_ReinforcementLearning}.


\subsubsection{The polynomial basis\ \cite{George_2011_AAAI_ValueApproxViaFourier}}
\begin{enumerate}
\item Given $d$ state variables $\mathbf{x}=\left[x_{1},x_{2},\cdots,x_{d}\right]\in\mathbb{R}^{1\times d}$,
the \textbf{\emph{simplest linear scheme}} uses \textbf{\emph{each
variable directly}} as a basis function \textcolor{blue}{along with}
a constant function, setting $\phi_{0}\left(\mathbf{x}\right)=1$
and $\phi_{i}\left(\mathbf{x}\right),\ 0\leq i\leq d.$ However, most
interesting value functions are too complex to be represented this
way.
\item This scheme was therefore generalized to the polynomial basis
\[
\phi_{i}\left(\mathbf{x}\right)=\prod_{j=1}^{d}x_{j}^{c_{i,j}}=x_{1}^{c_{i,1}}x_{2}^{c_{i,2}},\cdots,x_{d}^{c_{i,d}}\in\mathbb{R}^{\left(n+1\right)^{d}},
\]
where each $c_{i,j}$ is an integer between $0$ and $n$. We describe
such a basis as an order $n$ polynomial basis. For example, a 2nd
order polynomial basis defined over two state variables $x$ and $y$
would have feature vector:
\[
\Phi=\left[1,x,y,xy,x^{2},y^{2},x^{2}y,xy^{2},x^{2}y^{2}\right].
\]

\end{enumerate}

\subsubsection{Radial Basis Function\ \cite{George_2011_AAAI_ValueApproxViaFourier}}
\begin{enumerate}
\item RBF definition: another common common scheme for state vector $\mathbf{x}\in\mathbb{R}^{d}$;
the basic function is a Gaussian: 
\[
\phi_{i}\left(\mathbf{x}\right)=\frac{1}{\sqrt{2\pi\sigma^{2}}}\exp\left(-\frac{\|\mathbf{c}_{i}-\mathbf{x}\|_{2}^{2}}{2\sigma^{2}}\right),
\]
or a given collection of centers $\mathbf{c}_{i}$ and variance $\sigma^{2}$.

\begin{enumerate}
\item the centers $c_{i}$ are typically evenly along each dimension, leading
to $n^{d}$ centers for $d$ state variables and a\textbf{ }given
order $n$. 
\item $\sigma^{2}$ can be varied but is often set to $\frac{2}{n-1}$
\end{enumerate}
\item RBF's characteristics

\begin{enumerate}
\item RBFs only\textbf{\emph{ }}generalize locally—changes in one area of
the state space do not affect the entire state space. 
\item Thus, they are suitable for representing value functions that might
have discontinuities.
\item However, this limited generalization is often reflected in slow initial
performance. key\index{key}
\end{enumerate}
\end{enumerate}
\bibliographystyle{IEEEtran}
\phantomsection\addcontentsline{toc}{section}{\refname}\bibliography{F:/important/bibFiles/referenceBib}

\end{document}
